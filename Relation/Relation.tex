\documentclass[12pt, letterpaper]{article}
\usepackage[utf8]{inputenc}

\title{Relazione Progetto: XTetris}
\author{Alex Narder matricola:892300}
\date{\today}

\begin{document}
\maketitle

\section{Prefazione}

Questo progetto è stato realizzato in autonomia secondo lo standard ANSI C.
Il suo sviluppo è iniziato da subito e il codice è stato riscritto più di una volta nel 
corso del tempo.

\section{Struttura del progetto}

Il filesystem del progetto è semplice, nella directory principale sono presenti due directory;
Documents/ e Relation/. La directory Documents/ contiene la documentazione generata attraverso 
Doxygen, mentre nella directory Relation/ è presente la relazione del progetto in formato pdf scritta
tramite l'utilizzo di Latex. Inoltre nella directory pricipale troviamo i file sorgente, title.c contiene 
le funzioni grafiche dell'avvio quali la stampa del titolo e delle istruzioni, il file func.c contiene le 
istruzioni che poi verranno richiamate in ogni modalità. Le modalità vengono definite in mod.c,
al suo interno troviamo le funzioni che regolano l'andamento del gioco per modalità singleplayer e 
multiplayer, in fine vi è il file main.c ovvero il punto di partenza di tutto il gioco.
\newline
Per l'avvio del gioco è sufficente utilizzare il comando make, il Makefile compilerà i file 
in modo autonomo e poi avvierà il gioco.

\section{Funzioni principali}

Il progetto è strutturato in quattro file con estensione .c,
il file che contiene le funzioni essenziali si trova in func.c.
All'inizio del file si trovano le librerie usate, e subito dopo
troviamo le matrici dei tetramini. Per permettere il movimento del 
pezzo all'interno della tabella sono state utilizzate
due variabili (movx e movy) che vengono decrementate e incrementate a seconda degli
input ricevuti da tastiera, questa funzione è stata realizzata sia per il giocatore 
1 che per il giocatore 2, il giocatore 1 (protagonista anche della modalità singleplayer)
ha la funzione movep1() mentre il giocatore 2 movep2(), entrambe svolgono lo stesso compito.
Per evitare che i pezzi si scontrino con le pareti è stata creata la funzione 
otherblockcollision(), questa funzione viene anche impiegata per evitare che i 
pezzi si sovrappongano tra di loro.\\
I tetramini che vengono stampati nella tabella fanno parte di essa, il funzionamento del gioco 
si basa sulla sostituzione dei caratteri all'interno della matrice, questa sostituzione 
avviene attraverso la funzione copyblock(), che copia uno dei tetramini randomici all'interno di un tetramino generale (tetr1), e attraverso 
changetab(). Changetab() sostituisce i caratteri vuoti all'interno della tabella con i caratteri del tetramino,
la tabella viene successivamente ristampata con la funzione Tab() e i pezzi rimasti sulla tabella 
dalla scorsa stampa vengono cancellati utilizzando la funzione cleartab().\\
Nella modalità singleplayer viene stampata una sola tabella, mentre nella modalità multiplayer viene stampata sia la 
tabella del p1 che del p2 e i turni sono alternati attraverso uno scambio che avviene ogno volta che un tetramino tocca il suolo.
La funzione stoptetr() è quella che permette la collisione con il suolo
e con tutto ciò che è sotto il tetramino.\\ A questo punto ci si può muovere, ci sono le collisioni e i pezzi 
rimangono attaccati al suolo/altri tetramini. Per la rimozione delle righe utilizzo più di una funzione;
removelines() è la funzione principale che permette di rimuovere la riga piena e sostituirla con dei caratteri vuoti, 
poi questa funzione viene richiamata all'interno di checkscorep1() e checkscorep2(), rispettivamente per primo e secondo giocatore 
queste istruzioni abbassano di 1 riga tutto il campo partendo dalla riga vuota creata da removelines().
Ogni volta che la funzione rimuove delle righe un counter aumenta per verificare se più di una riga è stata
cancellata con un solo pezzo e in quel caso la funzione dupanother() conferisce i punti adeguati.
In conclusione per vincere in singleplayer è necessaria la funzione win(), appena tutti i pezzi danno somma 
0 il gioco si conclude e la partita è vinta dal giocatore. Nella modalità multiplayer i giocatori vincono in base al punteggio più alto, 
la vittoria viene determinata nel momento in cui i pezzi finiscono, oppure quando uno dei due giocatori sfora verso l'alto con un tetramino.
Quando uno dei tetramini, sia nella modalità singleplayer che multiplayer, esce dal campo viene dichiarato gameover, attraverso la funzione returngameover() per il singleplayer, e mod2gameover per il multiplayer.


\section{Difficoltà incontrate}

Le maggiori difficoltà incontrate sono principalmente legate ad alcune funzioni;
la funzione randomblock() mi ha dato molti problemi in quanto andava in contrasto con 
la decrementazione del numero di tetramini rimasti, ad ogni numero generato la funzione diminuiva 
il numero di tetramini rimasti fino ad andare in negativo, nonostante l'utilizzo di statement vari 
il problema non svaniva, quindi ho creato 3 funzioni che hanno reso più semplice il tutto, una funzione 
per il numero random, un check della quantità di tetramini e in fine una funzione che selezionava il tetramino.
Un problema che non sono riuscito a rivolvere è quello dell'ultimo tetramino, infatti 
una volta posizionato il penultimo tetramino il gioco finisce.
\\Slegato dai problemi delle funzioni è stato il "problema" dello standar ANSI C che non ha permesso 
l'utilizzo di alcune librerie grafiche molto interessanti. 

\end{document}
